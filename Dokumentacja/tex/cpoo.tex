\documentclass[a4paper,12pt,oneside,notitlepage,onecolumn]{article}

\usepackage{ucs}
\usepackage[utf8x]{inputenc}

\usepackage{fontenc}
\usepackage{graphicx}

\usepackage[OT4]{fontenc}
\usepackage[polish]{babel}
\usepackage{polski}
\usepackage{indentfirst}
\usepackage{graphics}

\usepackage[dvips]{hyperref}

\author{Michał Bobowski, Andrzej Dudziec}
\date{2013-03-16}
\title{Wektorowa filtracja medianowa - dokumentacja}

\begin{document}
  \maketitle
\section{Usuwanie szumu z obrazów}
Typowe metody akwizycji obrazów (np. matryca Bayera w aparacie fotograficznym) nie są odporne na pojawiające się losowo zakłócenia.
Implikuje to konieczność opracowywania algorytmów filtracji dolnoprzepustowej, czyli 'odszumiania' obrazów.

Najprostsze filtry opierające się na operacji splotu przestrzennego (konwolucji) posiadają znaczącą wadę, objawiającą się w rozmywaniu krawędzi obiektów występujących w obrazie.
Przyczyną tego zjawiska jest fakt, że natura szumu i krawędzi jest do siebie bardzo zbliżona – są to gwałtowne zmiany wartości pikseli, reprezentowane w widmie Fouriera jako składowe wysokoczęstotliwościowe.
Filtracja medianowa jest prostym sposobem na usunięcie szumu z obrazu, bez znaczącego obniżenia jakości krawędzi.
Dla każdego piksela wykonywane jest sortowanie wartości pikseli z sąsiedztwa, a element środkowy tego zbioru staje się nową wartością dla rozpatrywanych współrzędnych.

\section{Wektorowa filtracja medianowa}
Wektorowa filtracja medianowa jest rozwinięciem najprostszej wersji filtru dla obrazów barwnych.
Pikselem wstawianym do obrazu wynikowego nie jest w tym przypadku środkowy element okna analizy, ale piksel najmniej odległy od pozostałych.

Sąsiedztwo pomiędzy punktami może być zdefiniowane na różne sposoby. 
W naszej aplikacji zastosowaliśmy sąsiedztwo euklidesowe, czyli pierwiastek z sumy kwadratów różnic poszczególnych składowych.

\section{Adaptacyjna filtracja medianowa}
Pomimo stosunkowo niewielkich zniekształceń krawędzi, filtracja medianowa wprowadza pewną stratę informacji na obszarach nie będących zakłóceniami.
Adaptacyjny filtr medianowy ogranicza działanie filtru jedynie do pikseli, które znacznie różnią się od sąsiadów, a co za tym idzie z dużym prawdopodobieństwem są zakłóceniami.

Szczegółowy opis algorytmu:
\begin{enumerate}
 \item Zrób coś.
 \item A potem coś jeszcze.
 \item I delektuj się wynikiem.
\end{enumerate}


\section{Uogólniona filtracja medianowa}

\section{Wyniki testów i wnioski}
\subsection{Test 1}
\subsection{Test 2}
\subsection{Test 3}
\subsection{Test 4}
\subsection{Test 5}

\end{document}
